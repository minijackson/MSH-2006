\documentclass{report}

\usepackage[utf8]{inputenc}
\usepackage[T1]{fontenc}
\usepackage[french]{babel}
\usepackage{amsmath}
\usepackage{fullpage}
\usepackage{microtype}

\begin{document}

\section*{Catégories de manga en fonction du public visé}

\begin{description}
	\item[Kodomo:] Anime ou manga destiné aux jeunes enfants.
	\item[Shonen:] Anime ou manga destiné aux jeunes garçons (spécifiquement
		pour des personnes de moins de 14 ans mais qui peut avoir une audience
		beaucoup plus large).
	\item[Shojo:] Anime ou manga destiné aux jeunes filles.
	\item[Seinen:] Anime ou manga destiné aux hommes qui sont dans les environs
		de la 50aine.
	\item[Josei:] Anime ou manga destiné aux femmes allant de la fin de
		l'adolescence à la femme adulte.
	\item[Ecchi:] Anime, manga ou jeu érotique.
	\item[Hentai:] Anime, manga ou jeu pornographique.
\end{description}

\section*{Catégorie de manga en fonction de l'univers}

\begin{description}
	\item[Magical girl (Mah\=o Sh\=ojo):] Manga ou anime mettant en scène de
		jeunes filles ayant des pouvoir magiques, généralement de type sorcière
		ou magicienne (Par exemple: Madoka$\star$Magika, Sailor Moon).
	\item[Harem:] Manga ou anime ayant un personnage masculin proéminent pour
		un nombre plus important de personnages féminins principaux.
	\item[Mecha:] Manga ou anime mettant en scène des ``mecha'', à savoir des
		robots humanoïdes de taille gigantesque (Par exemple: Neon Genesis
		Evangelion, Goldorak)
\end{description}

\section*{Suffixes honorifique}

\begin{description}
	\item[San:] Niveau de politesse standard (équivalent à Mr./Mme.)
	\item[Kun:] Très souvent utilisé pour s'adresser à des garçons ou des amis
		masculins.
	\item[Chan:] Utilisé pour les bébés, jeunes filles ou jeunes garçons, ou
		pour les petits-amis/amis très proches.
	\item[Sensei:] Utilisé pour les professeurs, politiciens, docteurs ou
		autres personnes d'autorité.
	\item[Sama:] Utilisé pour les divinités et les personnes de la royauté.
	\item[Senpai:] Utilisé pour les supérieurs ou les personnes respectées.
\end{description}

\section*{Caractère type de personnages}

\begin{description}
	\item[Yandere:] Personnage psychologiquement instable ayant des
		sentiments pour un autre personnage mais utilisant ses pulsions
		meurtrières pour par exemple ``se débarrasser de la compétition''.
	\item[Tsundere:] Personnage dur et énervant au premier abord mais
		affectueux un fois sortie de sa coquille.
	\item[Kuudere:] Personnage froid et cynique en apparence mais attentionné
		en réalité.
	\item[Deredere:] Personnage adorable et énergique envers tout le monde.
	\item[Dandere:] Personnage calme est silencieux lorsque entouré par
		plusieurs personnes mais adorable et énergique lorsque seul avec une
		autre personne.
	\item[Kamidere, Himedere, Oujidere:] Personnage ayant un complexe
		de supériorité et se croient respectivement l'équivalent d'un
		Dieu (Kami), Princesse (Hime) ou Prince (\=Oji).
\end{description}

\section*{Style de dessins}

\begin{description}
	\item[Moe:] Style de dessin caractéristique et simpliste qui a pour but de
		faire sentir des sentiments d'affection envers les personnages (Par
		exemple: Nichijou).
\end{description}

\section*{Idéalisation}

\begin{description}
	\item[Bishonen:] Représentation de l'idéal masculin, en majorité un garçon
		efféminé, androgyne et jeune.
	\item[Bishojo:] Représentation de l'idéal féminin.
\end{description}

\end{document}

% vim: spell : spelllang=fr
