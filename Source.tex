\part{Des sources utiles}

\chapter{Blogs de français à l'étranger}
\section{blog 1:}


\chapter{Blogs d'étranger en France}
\section{nlog 1:}

\chapter{Sites sur l'interculturalité}
\section{site 1:}


\chapter{Nos sources pour ce dossier}
\section{A chacun sa culture qui fait rêver}
\subsection{Le Japon : Matsuris et Croyances}
\noindent
\url{http://www.japan-guide.com/e/e2063.html}\\
\url{http://fr.wikipedia.org/wiki/Matsuri}

\section{Les Otakus ou les passionnés de mangas}
\subsection{Histoire du manga}
\noindent
\url{http://www.france-jeunes.net/lire-le-manga-et-la-france-analyse-d-un-succes-25649.htm}\\
\url{http://hitek.fr/actualite/dossier-manga-naissance-arrivee-france_1924}

\subsection{Vocabulaire}
\noindent
\url{https://en.wikipedia.org/wiki/Glossary_of_anime_and_manga}\\
\url{https://www.asianfanfics.com/blog/view/411262/url}

\section{Concepts théoriques}
\subsection{4 Dimensions culturelles}
\noindent
\url{http://geerthofstede.nl/dimensions-of-national-cultures}\\
\url{http://news.telelangue.com/en/2011/09/cultural-theory}
