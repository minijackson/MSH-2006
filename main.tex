\documentclass[12pt]{book}

\usepackage[utf8]{inputenc}
\usepackage[T1]{fontenc}
\usepackage[francais]{babel}
\usepackage{fullpage}
\usepackage{pdfpages}
%\usepackage{microtype}
\usepackage{graphicx}
\usepackage[hidelinks]{hyperref}

\title{Explo}
\author{\'Elodie \textsc{Caroy}, Rémi \textsc{Nicole}, Jordan\textsc{Singkouson}, Timothée \textsc{Pallot}}
\date{}

\begin{document}
\begin{titlepage}
	\begin{sffamily}
	\includegraphics[scale=0.3]{esiee.jpg}
	\begin{center}

	%Titre
	\vspace*{2cm}
	{\huge \bfseries \emph{MSH-2006:}\\
	Exploration Interculturelle \\[0.4cm]}
	\begin{center}
	\Large Groupe Margaret Mead
	\end{center}
	
	
	\includegraphics[scale=0.8]{races-620x330.jpg}
	\newline
	
		 % Author and supervisor
    \begin{minipage}{0.4\textwidth}
      \begin{flushleft} \large
				\emph{Auteurs:}\\
        Elodie \textsc{caroy}\\
				Rémi \textsc{nicole}\\
				Timothée \textsc{pallot}\\
				Jordan \textsc{singkouson}
      \end{flushleft}
    \end{minipage}
    \begin{minipage}{0.4\textwidth}
      \begin{flushright} \large
        \emph{Responsable d'unité:}\\
				M. \textsc{markowski}\\
				
        %\emph{Chef d'équipe : } M. Chef \textsc{D’Équipe}
				\emph{9 Juin 2015}
      \end{flushright}
    \end{minipage} 
	\end{center}
	\end{sffamily}
\end{titlepage}

\textbf{Préface}

\paragraph{}
Ce livre a été rédigé par des élèves ingénieurs d’ESIEE Paris entre janvier et juin 2015.L’équipe Margaret Mead a fait ce travail dans le cadre d’un cours électif MSH2006 Explorations interculturelles. Il est assorti d’un blog ou d’un site web qui se trouve ici [mettez le lien svp].
\paragraph{}
Le cours existe depuis presque 30 ans.  En 2012 j’ai compris que la formule classique (cours magistral + travaux dirigés) était devenue obsolète.  En 2013le concept du livre interculturel a été lancé.  L’expérimentation a plu aux élèves.  Pour cette troisième série de livres, le cahier de charges a été renforcé. Des paliers de livrables ont été précisés. La version finale que vous avez entre vos mains a été précédée de sept autres
\paragraph{}
Les auteurs explorateurs de l’équipe Margaret Mead ont su produire un contenu intéressant et original. Ils ont également réalisé avec succès un projet sur la culture Otaku en France qui est beaucoup plus complexe qu’il ne parait.
\paragraph{}
Contrairement aux affiches des controverses qui ornent la grande rue de l’ESIEE et sont vues et lues par des milliers de visiteurs, cet objet est confidentiel et intime.  J’espère qu’il trouvera quelques lecteurs autres que mes collègues Christelle Fritz et Olivier Allard qui ont encadré et suivi avec moi la douzaine d’équipes.
\paragraph{}
Je remercie tous les auteurs pour leur candeur et spontanéité lors des explorations individuelles. Ce don de leur part explique (je pense) pourquoi ceci reste de loin mon enseignement préféré.
\paragraph{}
Krys Markowski\\4 juin 2015

\tableofcontents


\thispagestyle{plain}

\chapter*{Margaret Mead}
\addcontentsline{toc}{part}{Margaret Mead}

\begin{center}
	\includegraphics[scale=7]{MMead.jpg}
\end{center}

\paragraph{} Margaret MEAD est née le 16 décembre 1901 à Philadelphie d'un
professeur d'économie et d'une enseignante. Elle débute ses études à
l'université Pauw qu'elle quitte au bout d'un an pour étudier la psychologie à
l'université Barnard. Elle entre en 1923 dans le département anthropologique de
l'université Columbia. Elle obtient son doctorat en 1929 grâce à une thèse sur
la stabilité de la culture en Polynésie. Entre 1928 et 1935, elle fait de
nombreux voyages avec son troisième mari Reo Fortune. Elle donne naissance à sa
fille Mary Catherine Bateson Kassarjian qu'elle a eue avec son dernier mari
Gregory Bateson. Ses travaux portent notamment sur le rapport à la sexualité
dans les cultures traditionnelles de l'Océanie et du Sud-est asiatique. Elle a
notamment écrit ``le temps est venu, je crois, où nous devons reconnaître la
bisexualité comme une forme normale de comportement humain'' et ``un grand
nombre d'êtres humains - probablement la majorité - sont bisexuels en ce qui
concerne leur capacité à éprouver des sentiments amoureux''. Elle est morte des
suites d'un cancer le 15 novembre 1978. Elle a reçu la Médaille Présidentielle
de la Liberté à titre posthume en 1979. Elle est inscrite au National Women's
Hall of Fame.

\part{Explorations}
\chapter{Trois Cas Observes}
\paragraph{}
Le XXIe siècle et ses innombrables moyens de communication ont rendu les
situations interculturelles de plus en plus courantes. C’est pourquoi le choc
des cultures est une expérience que chaque Français est amené à vivre
quotidiennement. Cette expérience peut émerveiller, interpeller ou choquer
comme vu ci-dessous.

\section{Emerveillement}

\subsection{1\ier version}
\paragraph{}
En France peu de tradition ont perduré au cours du temps. En effet, la France
étant un pays issue de nombreux brassage, les habitudes et les traditions sont
propres à chaque famille. Au contraire, le Japon, bien qu’étant très avancé au
niveau des nouvelles technologies, a su garder nombre de ses traditions tel que
le fait d’aller au temple prier pour la nouvelle année. Cela m’étonne toujours
avec plaisir de penser à ce contraste qui cohabite entre la culture
traditionnelle et la culture moderne japonaise.

\subsection{2\ieme version}

\paragraph{}
\begin{center}
\includegraphics[scale=0.7]{Amsterdam1.jpg}
\end{center}

Parmi les situations qui m'a le plus émerveillé, c'est bien le jour où j'ai vu
deux sans abris aux alentours d'Amsterdam.  Nous étions la veille de noël et
j'ai vu ces deux personnes par terre se raconter des histoires et rigoler
ensemble. Cette scène m'a paru inespérée car il faisait froid, mais ils avaient
l'air si heureux.  Cette scène m'a fait rendre compte que nous devons profiter
de chaque moment de notre vie et que dans toutes les situations, nous devons
savoir rire et être heureux.

\paragraph{}
\begin{center}
\includegraphics[scale=0.7]{Amsterdam2.jpg}
\end{center}

De nos jours, de grandes inégalités sociales ont vues le jour et nous devons
les combattre pour que personne ne meurt de froid ou de faim. Je me rends
compte que le bonheur ne vient pas seul, il faut porter de l'attention à
quiconque, car nous partageons cette même planète.  Cette expérience de vie m'a
fait comprendre beaucoup sur l'homme et sur ce dont il a besoin. Nous devons
être heureux ensemble, apprenons à vivre tous ensemble en s'entraidant.

\subsection{3\ieme version}
\paragraph{}
Des contacts interculturels qui m'ont émerveillés et qui continuent encore de
m'émerveiller sont le fait que les différences culturelles sur internet dans le
domaine de l'informatique sont complètement ignorées. En effet, lorsque le
sujet est différent de l'interculturalité les personnes sur internet parlent en
général librement sans prendre considération de l'origine ou de la culture de
l'autre. J'en conclue donc que de par le fait qu'ils n'aient pas de contacts
visuels directs avec les personnes, ils ne réfléchissent pas à l'origine ou à la
culture de l'autre, ce qui m'incline à penser que le principal facteur des
dissonances interculturelles est le contact visuel.

\subsection{4\ieme version}
\paragraph{}
Ayant peu voyagé, je n'ai pas pu connaître beaucoup de situations
interculturelles fortes. Cependant, un exemple récent me vient en tête. J'ai
suivi à la télévision et sur Internet les images des manifestations pour la
paix du 11 janvier 2015 à Paris et dans les grandes villes françaises. Alors
que le clivage entre les cultures me semblait se développer dans notre pays, et
que le « creuset français » (Gérard Noiriel) ne semblait plus une réalité,
cette manifestation m'a montré qu'une France hétérogène – dans ses origines –
mais uniforme – dans ses idéaux – existait toujours. Certes, nombreux étaient
ceux qui ont participé à ces marches pour prouver ou revendiquer une position,
comme les nombreux chefs d'État présents, ou les représentants religieux venus
dénouer les ambiguïtés entre religion et obscurantisme. Cependant, l'image
d'une foule issue de toutes les cultures et toutes les religions arborant les
couleurs françaises ou même celles de Charlie Hebdo, un journal plutôt connu
pour diviser que rassembler, était un symbole fort. Des banderoles de la
citation attribuée à Voltaire par une biographe anglaise s’étalaient en
plusieurs langues : « I disapprove of what you say, but I will defend to the
death your right to say it ». Cela peut sembler mièvre et patriotique, mais il
semblait se dégager un « esprit français ».

\begin{center}
\includegraphics[scale=0.5]{charlie.jpg}
\end{center}

\section{Interpellation}
\subsection{1\ier version}
\paragraph{}
J’ai participé à trois voyages scolaires durant mes années lycée. Lors des
voyages en Irlande du Nord et en Angleterre, nous avons séjourné dans des
familles d’accueil. Lors du voyage en Angleterre, je n’ai pu apercevoir qu’une
seule fois le mari et les enfants. En effet, seule la maitresse de maison
s’occupait de nous et nous accordait du temps. Cependant, bien qu’elle fût
accueillante, nous prenions nos repas entre nous sans la famille. C’est ainsi
que nous avons passé ces quelques jours dans cette famille sans beaucoup de
contact avec celle-ci. Au contraire, lors du voyage en Irlande, nous prenions
tous nos repas avec les parents de la famille. En effet, les enfants ayant un
rythme de vie moins souple mangeaient aux heures habituelles pour leur culture
tandis que les parents nous attendaient pour manger avec nous lorsque nous
rentrions le soir. Cependant, nous avons pu voir certain des enfants le soir
qui venaient discuter avec nous et les parents pendant qu’on prenait notre
repas et eux se contentaient d’un thé. Ainsi, nous avons pu discuter des
différences de culture entre les Irlandais et les étudiants qu’ils
accueillaient, de leur histoire (ce pourquoi nous faisions ce voyage scolaire),
… Je fus donc interpellé par la différence d’accueil entre les Anglais et les
Irlandais. Bien que chacune des familles étaient payée par une association pour
nous accueillir, le ressenti ne fut pas le même : les Irlandais nous
accueillait aussi par envie et pas seulement pour l’argent.

\subsection{2\ieme version}
\paragraph{}
\begin{center}
\includegraphics[scale=0.7]{Afrique.jpg}
\end{center}
Ce qui m'a émerveillé, c'est de voir ma cousine partir en Afrique, plus
précisément au Liberia, pour combattre une maladie qui a fait énormément de
mort, appelé Ébola.  Travaillant dans l'humanitaire, elle m'a toujours
passionné par tout ce qu'elle a accomplie jusqu'à présent. Ces derniers temps,
j'ai été fasciné par son courage d'entreprendre les choses et son désir de
vouloir aider ces personnes, souffrant de l’Ébola.

\paragraph{}
\begin{center}
\includegraphics[scale=0.7]{Afrique2.jpg}
\end{center}
Actuellement, elle est une des managers d'une ONG en santé publique, permettant
de limiter la contamination du virus. Elle me raconte par mails ses journées et
comment est-ce qu'elle fait pour ne pas craquer émotionnellement. Ses
différentes expériences à travers le monde m'ont donné l'envie de voyager et
d'aider les personnes qui m'entourent.

\subsection{3\ieme version}
\paragraph{}
Ce qui m'a interpellé lors d'un voyage en Allemagne est le fait que les
allemands mangent de la charcuterie au petit déjeuner. Après des études
approfondies, j'ai jugé que ce rituel culturel méritait sa place en Allemagne
et sûrement dans d'autres pays. C'est du moins ce que mes papilles gustatives
en ont déduit.

\subsection{4\ieme version}
\paragraph{}
Il y a près de quatre ans, je suis parti en Angleterre pendant deux semaines
avec un ami. Ses parents l’avaient poussé à partir pour un séjour linguistique
en groupe, et je l’ai accompagné. Je me suis donc retrouvé avec lui en famille
d’accueil, et je pensais que nous serions souvent en contact avec les « natifs
». J’ai assez vite déchanté : les familles d’accueil sont une industrie
anglaise ! Dans cette petite maison, nous étions quatre étrangers d’origines
différentes, mais la famille habituée ne jouait pas la carte du melting-pot :
nous étions dans des chambres différentes, nous n’étions pas soumis aux mêmes
horaires, nous avions rempli un formulaire sur les plats que nous n’aimions
pas… De même, lors des sorties organisées par le groupe, et lors des cours
d’anglais organisés par un professeur non francophone, les jeunes français (qui
pourtant ne se connaissaient pas) s’isolaient pour discuter entre eux. Cela
peut sembler naturel : timidité, hésitation à parler en anglais… Mais les
organisateurs ne faisaient rien contre, que ce soient les Français tout aussi
hésitants ou les Anglais habitués à ce genre de situation. Finalement, je n’ai
pas progressé en anglais au cours de ce voyage, et n’ai connu aucune situation
interculturelle. Bien que timide de nature, j’en étais très déçu.


\section{Choc}
\subsection{1\ier version}
\paragraph{}
Un été je suis partie une semaine en Toscane. Ne pouvant pas marcher longtemps
suite à de nombreuses entorses, nous avons fait le tour de la Toscane en
voiture en nous arrêtant dans les villes plus ou moins touristique sur notre
chemin (Florence, Pise, …).  En arrivant sur Florence, nous avons pris une
route à deux voies. Nous avons vu devant nous trois voitures côtes à côtes.
Cela m’a choqué de voir ce manque de respect pour le code de la route et ce
mépris du danger. En effet, les Français ne sont pas toujours les meilleurs
dans le respect du code de la route mais ça n’en devient pas un sport national
comme en Italie : nous avons vu de nombreuses absences de respect du code de la
route.

\subsection{2\ieme version}
\paragraph{}
\begin{center}
\includegraphics[scale=0.7]{Thai1.jpg}
\end{center}

Lors d'un voyage à Thaïlande, j'ai décidé de découvrir la boxe thaïlandaise,
appelé plus communément « Muay Thai » dans le pays de provenance de cet art
martial. La boxe thaïlandaise se base sur quatre techniques fondamentales :
les coups de poings, pieds, genoux et coudes. Les boxeurs portent des gants et
un short afin de faciliter les mouvements des jambes. Les combats s'effectuent
en 5 reprises de 3 minutes entrecoupées de pause de 2 minutes. Le tout se
déroule sur une musique de fond thaïlandaise très envoutante.  Quelques jours
après mon arrivée, je me baladais dans les rues du quartier de Sukhumvit, dans
des avenues très vivantes, jours et nuits sur toute leur longueur, je me suis
dirigé vers un bâtiment qui ressemblait à un temple. Tout le monde criait et
voulait y rentrer, j'ai convaincu les amis avec qui j'étais d'y aller, pour
voir de quoi est-ce qu'il s'agissait.  L'entrée était gratuite, mais il y avait
des guichets où nous pouvions miser de l'argent sur des numéros. Ces numéros
étaient tout simplement des boxeurs et nous étions arrivés en face d'un ring de
boxe autour duquel se trouvait une foule incroyable.

\begin{center}
\includegraphics[scale=0.7]{Thai2.jpg}
\end{center}

\paragraph{}
Sur le ring, se trouvait deux enfants âgés de 8 ans combattant l'un contre
l'autre sous les applaudissements et les encouragements d'adultes qui criait
victoire pour la personne sur laquelle il avait misé. Cette scène eut l'effet
d'un tonnerre qui s'abattit sur moi, car je ne pouvais rien à faire pour arrêter
ce massacre.  Malheureusement, j'ai demandé aux spectateurs la raison pour
laquelle ils se battaient et on me répondit « de baht ! », qui veut dire « pour
l'argent » en thaïlandais. Totalement outragé de voir ce qu'il se passait en
Thaïlande, j'ai décidé de retrouver le jeune boxeur ayant perdu et n'ayant pas
gagné d'argent, pour le féliciter et lui donner 400 bahts, qui est l'équivalent
de 10 euros, mais qui représente une somme énorme pour eux.  Ces découvertes
m'ont mis totalement mal à l'aise et j'ai décidé de combattre cette cause en
revenant à la source, la pauvreté.

\begin{center}
\includegraphics[scale=0.7]{Thai3.jpg}
\end{center}


\subsection{3\ieme version}
\paragraph{}
Une situation qui m'a choqué en étant en voyage au Sri-Lanka est la conduite de
ces derniers. En effet, il m'est arrivé d'être à quatre voitures sur une route
à deux voies à sens alternés. J'en ai donc déduit que les Sri-Lankais se
faisaient confiance au point d'avoir ce genre d' « interactions » sur la route.

\subsection{4\ieme version}
\paragraph{}
Ma famille est très attachée au patrimoine : mon père et ma mère sont des
spécialistes des monuments historiques, mon grand-père et mon oncle sont
antiquaires. J’ai donc été élevé dans cette ambiance, ne partant en voyage que
pour visiter des églises. C’est pourquoi, lors d’un séjour à Venise où, étant
très jeune, je ne prêtais pas grande attention à la culture locale, j’ai été
frappé par l’approche différente que l’Italie a de la culture. En effet, de
nombreuses églises, du fait du nombre de touristes, font payer les visiteurs à
l’entrée ! Cela me semblait inconcevable, d’autant que je n’étais pas certain
que cet argent irait à la paroisse ! Plus tard, lors d’un voyage scolaire à
Pompéi, j’ai remarqué l’état de délabrement de la cité millénaire, qui est dû à
l’absence de moyens de conservation plus qu’aux dégâts infligés par le Vésuve.
Plusieurs bâtisses se sont effondrées, redressées à la va-vite avec du béton et
cachées à la vue des visiteurs par des pancartes. Un bâtiment du forum a même
été transformé en cafétéria. Ma mère m’expliqua que l’argent des visiteurs
n’allait pas à la conservation du site mais plus ou moins à la pègre locale
(n’oublions pas qu’il s’agit de Naples !). Si l’Italie comme la France jouit
d’un patrimoine exceptionnel, l’approche italienne est beaucoup plus économique
! J’en étais d’autant plus attristé qu’avec Pompéi peut disparaître une portion
d’histoire.

\begin{center}
\includegraphics[scale=0.5]{Pompei.jpg}
\end{center}


\chapter{Bienvenue au café}
\paragraph{}
Chaque café possède sa propre vie, sa propre ambiance qui varie suivant
l’heure, le jour et leur voisinage. Du café de la gare peuplé de voyageurs au
comptoir montagnard abritant les skieurs frigorifiés, en faisant un détour par
Londres, nous allons vous donner un aperçu de l’atmosphère typique de ces lieux
de rencontres.

\section{1\ier version}
\paragraph{}
La vie des cafés varie grandement suivant leur situation géographique, la météo
et la période.

\paragraph{}
Prenons deux cafés dans une même station de ski familiale dans le massif
central : la brasserie des pistes, vu entre 18h45 et 19h30, fait
bar/brasserie/restaurant ; le polar beer, vu entre 16h et 17h, fait bar toute
la journée et snack le midi. Il se situe juste en bas des pistes.

\begin{center}
\includegraphics[scale=0.15]{brasserieDesPistes.jpg}
\includegraphics[scale=0.15]{PolarBeer.jpg}
\end{center}

\paragraph{}
À la brasserie des pistes, il y avait le patron, la patronne au bar, le
cuistot, le pizzaiolo, et deux serveurs. Au contraire, au polar beer, il n’y
avait que deux serveuses. Cette différence s’explique par le fonctionnement de
chacun des établissements. En effet, étant servi à table à la brasserie des
pistes, il est nécessaire d’avoir plus de personnel qu’au polar beer où les
clients doivent aller chercher leurs consommations au comptoir. De plus, dans
chaque café la consommation est payée au moment de la prise en main de
celle-ci.

\paragraph{}
Dans chacun des établissements il y avait une télévision. A la brasserie des
pistes, la télévision permet d’écouter RFM TV sauf au moment du journal
télévisé régional où l’on bascule sur France 3. Au contraire, au polar beer, la
télévision montrait des surfeurs et il y avait de la musique lounge en fond
sonore. On pouvait cependant remarquer que c’était surtout les adolescents et
jeunes adultes masculins qui regarder la télévision au polar beer quand
personnes ne regardait la télévision à la brasserie des pistes.

\paragraph{}
La décoration de la brasserie des pistes était basique : des tables carrées
pour la partie restaurant et quelques tables rondes pour la partie bar, des
chaises en fer, quelques banquettes. Au contraire, au polar beer, la décoration
était moderne et humoristique : des tables hautes, longues ou rondes avec des
chaises hautes, les murs étaient principalement rouges avec des jeux de mots
par-ci, par là tels que « beer crossing » d’un côté du mur et « bear crossing »
de l’autre. De plus, le comptoir était séparé en deux parties distinctes : le
bar où des bouteilles étaient exposées juste derrière et une partie snack
servant uniquement les boissons chaudes, crêpes et gâteaux à cette heure-ci.

\begin{center}
\includegraphics[scale=0.15]{PolarComptoir.jpg}
\includegraphics[scale=0.15]{PolarDeco.jpg}
\end{center}

\paragraph{}
La population de chaque café était également différente.

\paragraph{}
En effet, à la brasserie des pistes, il n’y avait personnes entre 18h45 et
19h00. Puis deux types de personnes sont arrivés peu à peu : il y avait les
familles avec enfants qui venaient diner et deux duos qui venaient boire un
coup. L’un des duos était des hommes d’environ trente ans qui ont pris chacun
deux vins chauds en jouant aux cartes. Le deuxième duo était un couple ayant
environ la soixantaine qui a également pris un vin chaud par personne. Ce choix
de boisson s’explique par la situation géographique du café : c’est une boisson
qu’on boit souvent au ski. Vers 19h30, la plupart des tables du restaurant
étaient prises par les familles avec enfants voulant diner. De plus, j’ai pu
remarquer que le comportement des serveurs variait avec les enfants : ils
prenaient un ton paternaliste et était aux petits soins. Enfin comme il
pleuvait et neigeait, la terrasse était vide.

\paragraph{}
Au polar beer, la population était constituée de skieurs. Vers 16h, il
s’agissait principalement de famille tandis que vers 16h45, la population
arrivante était principalement des groupes de jeunes adultes. Certains
restaient avec leur manteau et repartait après avoir bu leur consommation et
s’être un peu réchauffé. D’autres se déshabillaient un peu plus et restaient
plus longtemps. La principale consommation était des chocolats chauds même si
certains adolescents buvaient des sodas.  Certains sodas et la bière étaient
d’origine locale tel que l’ « auvergnat cola ». Enfin, le soleil étant au
rendez-vous bien qu’un peu couvert, les fumeurs consommaient leurs boissons et
crêpes sur la terrasse tout en fumant.

\begin{center}
\includegraphics[scale=0.15,angle=270]{AuvergnatCola.jpg}
\end{center}

\paragraph{}
De plus, si on compare la vie de ces deux cafés à celle d’un café de campagne,
on se rend compte que c’est encore différent.

\paragraph{}
En effet, le vendredi, jour de marché, de 11h à 11h30 en août à Arcis sur Aube,
petite ville de 3000 habitants, le café et sa terrasse sont bondés. Cette
densité de population au sein de café est due au fait que les habitués s’y
retrouvent pour discuter des dernières nouvelles. Mon grand-père étant un
habitué, il s’arrête à chaque table pour dire bonjour aux gens qu’il connait.
Puis il entre dans le café et dire bonjour au patron tout en commandant un
café. Ensuite, il va s’assoir avec une de ces connaissances, en l’occurrence
une grand-mère et ces trois petits enfants. La serveuse apporte le café. Les
petits enfants jouent au babyfoot pendant que les grands-parents discutent des
dernières nouvelles. Les personnes travaillant au café sont la serveuse, le
patron et son fils qui semble aider les jours d’affluence.

\begin{center}
\includegraphics[scale=0.5]{BarArcis.PNG}
\end{center}

\section{2\ieme version}
\paragraph{}
\emph{Adams Café, Hammersmith, London :}

\paragraph{}
Le premier jour de printemps était arrivé, mais le froid était toujours au
rendez-vous. Près de la station Hammersmith, un quartier très attractif du
grand Londres, je me suis replié auprès du café Adams pour me réchauffer.

\paragraph{}
Très peu de temps après mon arrivée, une serveuse est venue vers moi pour me
proposer de boire un verre et de m’asseoir près de la grande baie vitrée du
café. Je lui répondis que je désire prendre un café avec un pancake si cela
était possible. Le café était très chic et décoré d’un bois rustique qui me
rappelait la bonne vieille époque.

\paragraph{}
Je me suis donc assis à l’endroit qu’elle m’avait proposé, je pouvais voir ces
personnes qui couraient près du « Tube Station » de la station d’Hammersmith,
très utilisé de jour comme de nuit. Dans le café, beaucoup de personne s’y
trouvait, dont un couple de personne âgé qui m’a interpellé accompagné de leur
petit fils. Ils rigolaient et semblaient heureux d’être ensemble.

\paragraph{}
Au fond du café, un homme tout seul se trouvait là, je me demandais ce qu’il
faisait et pourquoi est-ce qu’il préférait être là, plutôt que chez lui.
Réfléchissait-il à ce qu’il voulait faire dans la journée ?

\paragraph{}
Alors que j’observais les clients autour de moi, la serveuse m’interrompit en
me pausant mon café sur table en me souhaitant « Have a good lunch ».  Je la
remerciais en la saluant pour lui dire qu’elle pouvait disposer, car je n’avais
besoin de rien d’autre que d’être seul.

\paragraph{}
Pendant ce temps, un jeune couple entrèrent dans le café, avec des sacs pleins
les mains et des bagages sur le dos. À l’heure accent, j’en étais sûr, c’était
des Français. Je fus ravi qu’ils s’assoient à côté de moi, car je me suis dit
que je pouvais comprendre ce qu’ils disaient, sans même qu’ils ne sachent que
je comprenne ce qu’ils disaient.

\paragraph{}
Ces dernières racontaient leur premier ressentis de leur voyage à Londres et
semblaient véritablement conquis par l’ambiance de ces lieux.

\paragraph{}
Je déposais l’argent sur la table pour payer ce que j’avais commandé, avec ce
qu’on appelle un « tips », qui est un pourboire de 2 euros que j’ai laissé pour
cette charmante serveuse.

\paragraph{}
Avant de partir, j’ai salué les 2 Français « Have a nice day, dear countryman »
et ils ont souris, ce qui m’a fait chaud au cœur, car moi aussi j’étais heureux
de me trouver à Londres.

\section{3\ieme version}

\paragraph{}
\emph{18 février, Le Perrier, Châtel}

\subsubsection{Introduction}
\paragraph{}
J'ai fait mes observations dans le café de Châtel, pendant mes vacances de ski.
Ainsi, il faut garder en mémoire que l'esprit des personnes présentes dans le
café est plus centré sur les vacances et donc l'ambiance est très probablement
plus joviale que la ``normale''.

\subsubsection{Différences ethniques}
\paragraph{}
J'ai pu remarqué que les groupes entrant dans le café ne possédaient pas
beaucoup de différences ethniques. On peut donc supposer que les groupes
partant en vacances de ski (la plupart de manière familiale par suppositions)
le font la plupart du temps avec des personnes de la même ethnie.

\subsubsection{Consommations}
\paragraph{}
Parmi les boissons consommées dans différents groupes, il existe beaucoup de
similitudes entre les consommations de chaque groupe. Par exemple, un groupe
entier consommera plus des bières en général alors qu'un autre groupe
consommera des boissons plus sucrées/chaudes comme des chocolats chaud ou des
cafés.

\subsubsection{Conversations}
\paragraph{}
Pour des raisons de vie privées, je n'ai pas écouté avec précision les
conversations des personnes. Cependant avec des mots clefs, j'ai pu déduire
que les conversations tournaient en grande majorité autours du ski même
(performances, que faire ensuite). Sinon, les discussions tournaient souvent
autour des vacances.

\subsubsection{Positions}
\paragraph{}
Des comportements récurrents que j'ai pu observer étaient que la plupart du
temps les hommes dans la tranche d'âge 20-40 se mettent souvent ou presque
toujours dans les coins ou dos à un mur, laissant aux enfants des places avec
chaises du côté où d'autres personnes, étrangères, passeraient.

\section{4\ieme version}
\paragraph{}
\emph{Jeudi 12 février, 8h25 – Café Cours des Roches, Noisiel}

\paragraph{}
Je ne me sens pas à ma place, et en même temps je n’ai pas envie de bouger. Un
coup d’œil rapide à ma montre, huit heures vingt-cinq – trois minutes, c’est
bon, je peux attaquer mon chocolat chaud. Il n’y a rien de pire que de se
brûler la langue dans la précipitation. Mon croissant est déjà bien entamé,
mais de toute façon, je n’oserais pas le tremper dans ma tasse ; je n’ai jamais
su si c’était bien vu en public.

\paragraph{}
Il y a dix minutes, je m’installais au café, commandais chocolat, jus et
viennoiserie, avec un léger trac. L’objectif : s’asseoir et observer. Pas
facile, quand on se sent soi-même en terre inconnue : je n’ai pas l’habitude
des cafés, et eux-mêmes n’ont sans doute pas coutume de voir un étudiant de 19
ans – moins en apparence – tout seul, balayer la salle un bloc-notes à la main.

\paragraph{}
La salle est calme, peu remplie, les clients vont et viennent. La plupart ne
restent pas longtemps : nous sommes en face de la gare, et près de toutes
sortes de bureaux : trésor public, banque populaire... On vient au café des
Roches pour y déjeuner sur le pouce, en attendant ou en sortant du train. Ou
peut-être de ce cortège de bus dont les horaires ne se marient jamais à ceux du
RER. Je me félicite de ne pas être venu un jour de marché : la place et les
trottoirs y sont habituellement noirs de monde, alors je n’ose imaginer
l’atmosphère du bar.

\paragraph{}
Je remarque tout de même des profils différents du cadre de passage et du
voyageur affamé. Au comptoir, quelques habitués venus partager les dernières
nouvelles avec le patron. J’y associe sans doute trop vite le cliché habituel ;
après tout, rien ne me dit qu’il s’agit du patron, d’ailleurs je ne vois aucun
torchon sur son avant-bras. Je ne saurais nommer le contenu de leurs verres,
mais il me semble plus éthylique que mon modeste jus d’orange. Plus loin, à une
table, un jeune couple sûrement trop tôt réveillé. Difficile d’évaluer leur
profession. Aspirés par leur conversation, ils délaissent leurs boissons
chaudes. Dans quel monde vivons-nous ?

\paragraph{}
Les quelques places en terrasse – un joli mot pour désigner le trottoir – sont
délaissées, sans surprise par ce froid.  J’aperçois un groupe de trois
lycéennes s’y asseoir, puis se raviser et entrer dans le bar. On leur apporte
vite des cafés. Rien d’étonnant, le lycée Gérard de Nerval est à deux pas. Je
me surprends à penser qu’elles sont un peu jeunes pour ce genre
d’établissement, avant d’estimer que je n’avais que deux ou trois ans de plus
qu’elles. On prend souvent un café parce que c’est une boisson de grand, et on
est trahi par la quantité de sucre, de lait et de mousse saupoudrée de cacao
qu’on y instille. Ça ne manque pas ici.

\paragraph{}
J’en arrive à une conclusion : un café, c’est le lieu des grands par
excellence. Le lieu de ceux qui ont renoncé à un peu de sommeil supplémentaire
pour venir partager un verre, une tasse, un gobelet entre collègues. La société
plutôt que le rêve. Ce qui explique sans doute pourquoi je ne m’y sentais pas à
mon aise, et qu’un quart d’heure a suffi à me faire penser que ces demoiselles
devraient plutôt être en cours à huit heures et demie. Une vraie réflexion de
grand – c’est-à-dire de vieux.

\chapter{Communication Homme/Femme}
\paragraph{}
bla

\section{1\ier version}
\subsection{Hommes}
\paragraph{}
Les hommes sont hautement engagés dans la conversation, ils parlent de leurs vies et de celle des autres dans
les lieux qu’ils apprécient et où ils y vont régulièrement. Leurs sujets de conversations sont généralement les
mêmes, on peut y retrouver leurs histoires personnelles avec leurs femmes, ce qu’ils se passent à leurs lieux
de travail ou leurs désires. Dans leurs conversations, beaucoup de sujet sensible qu'ils ne préfèrent pas que
les femmes entendent, parce-qu'il s'agit de chose dont on ne doit pas en parler, par pudeur vis-à-vis des
femmes. C'est ce qu'on appelle « parler de choses tabou ».

\paragraph{}
Concernant leurs façon d’interagir et de converser entre eux, on peut remarquer qu'ils sont très réactif face à
toutes leurs remarques et n’hésitent pas à s’imposer face aux autres. Leurs mécontentements face à une
remarque pour montrer leurs mécontentements peut prendre des proportions disproportionné. Il n'y a aucune
diplomatie dans leurs conversations.

\paragraph{}
On peut comprendre qu'ils sont dans une logique de supériorité et qu'ils cherchent toujours à être au-dessus
des autres. En effet, les hommes sont sensibles dans les interactions à créer une dynamique de pouvoir et ils
parleront de façon à permettre une différenciation entre les dominants et les dominés.

\paragraph{}
Ces discussions ont eu lieu au Mac Donald's avec un groupe d'amis, un vendredi soir après le boulot pour
certains et les cours pour d'autres. Ces petits moments de retrouvaille sont hebdomadaire et font partie de
notre routine.

\paragraph{}
Par rapport aux conclusions de Deborah Tannen, je trouve que les hommes cachent soigneusement leurs
doutes. En effet, ils considèrent que c'est se mettre dans une position de faiblesse et seront toujours attentif à
ne pas se mettre en situation d'infériorité.

\subsection{Femmes}
\paragraph{}
Les femmes sont hautement respectueuse du locuteur. Les femmes conversent de leurs moeurs et des
conduites individuelles. Ces sujets sont très intéressant pour ces personnes, car elles sont très curieuse de
découvrir la vie de leurs amies. Les histoires d'amours, les livres qu'elles ont lu récemment, les films et séries
qu'elles ont adorés, beaucoup de sujet de conversation sont à noter de leurs côtés.

\paragraph{}
Bien que les conversations sont denses et sans arrêt, ce qui me surprend c'est que personne ne se coupe la
parole et tout les personnes respectent l'opinion du locuteur. Dans les rituels de conversation des femmes, les
femmes utilisent beaucoup plus fréquemment les formules de politesse et d'excuse que les hommes comme
« Je suis désolée ». Parmi ce rituel, il y a aussi l'échange de compliment que les femmes se font de façon
réciproque.

\paragraph{}
C'est une façon de continuer le dialogue pour ne pas laisser ce qu'on appelle « un blanc », en minimisant ses
propres qualités sur un mode qui implique que l'autre en reconnaîtra le caractère et aura la possibilité de lui
adresser une compliment en retour.

\paragraph{}
C'est discusion ont lieu au Starbuck avec le groupe d'amie qu'elles gardent depuis les années du lycée.
Comme pour les hommes, c'est en soirée le vendredi soir, la veille du week-end. Ce sont aussi des
retrouvailles hebdomadaire.

\paragraph{}
Par rapport aux conclusions de Deborah Tannen, les femmes feront attention à l'autre concernant la
domination. En effet, les femmes apprennent à minimiser les signes qui indiquent la supériorité de l'une sur
les autres. Elles prônent l'égalité entre elles.

\subsection{Mixtes}
\paragraph{}
Bien que nous avons remarqué que les hommes et les femmes avaient différente culture de communication,
jusqu'au point de qualifié ces conversations de « multiculturelle », on remarque qu'on peut parler de la même
chose que dans les groupes composé uniquement d'homme ou femme, mais il y a des répercussions sur les
jugements de compétence, de confiance en soi qui sont portés sur les personnes. En effet, ce que nous
apprenons par la culture des autres nous apprend à juger et à évaluer les autres par rapport aux autres.
Notamment par la directivité du message, les temps de pause et le choix des mots.

\paragraph{}
Les conséquences sont que les femmes vont dire « nous » et les hommes employer le pronom « je ». La
principale problématique à ce jours, dans le monde du travail, c'est la prédominance de certains face à
d'autres. C'est la raison même de la venue du sexiste, car chaque personne a été catégorisé et de nombreux
mouvement tente de mettre fin à cette complication dans la sphère du travail.

\paragraph{}
Les femmes et les hommes sont bien égaux, ce n'est que leurs cultures et façon de faire qui sont différente.

\section{2\ieme version}
\paragraph{}
Dans chaque cas, je suis restée à distance en position d’observatrice.
\subsection{Hommes}
\paragraph{}
Les deux groupes masculins ont été observés dans le rer, le premier, samedi à 17h45, et le deuxième, dimanche à 21h.
\subsubsection{groupe A}
\paragraph{}
	Il s’agissait d’un groupe de trois personnes où principalement deux des trois participait à la discussion. Lorsqu’ils parlaient des drones, ils se sont penchés sur un téléphone portable pour illustrer leur discussion. Puis ils ont parlé d’un sport tout en faisant des imitations pour illustrer leurs propos. L’un d’entre eux a souri alors qu’il parlait d’un partiel qu’il a tout juste réussi ce qui l’a embêté. Ensuite ils ont discuté des difficultés qu’ils rencontreront l’année prochaine dans leurs études. Lors de cette discussion, l’un d’eux à donner une tape sur l’épaule de l’autre pour lui montrer son soutien. Durant toute leur discussion, ils se sont rarement regardé les uns les autres.
\subsubsection{groupe B}
\paragraph{}
	J’ai pu observer dans le rer deux hommes qui se sont croisés par hasard. Ces deux hommes se connaissaient de longues date mais ne s’étaient pas vu depuis un moment. Ils se sont serrés la main et ont échangé les dernières nouvelles. Durant cette discussion, ils ne se sont que très peu regardé comme le groupe précédent.
	
\subsection{Femmes}
\paragraph{}
Pour les groupes féminins, l’observation a eu lieu dans un bar entre 21h et 22h.
\subsubsection{groupe A}
\paragraph{}
	Il s’agit de deux filles. L’une se plaint d’un homme tandis que l’autre l’écoute et lui donnes des conseils de temps à autres. La fille qui est écoute est adossée à son dossier tandis que l’autre est penché sur la table. De plus, la fille qui se plaint joue avec des objets tels que son bracelet ou un élastique. La plainte de la fille est ponctuée de gestes amples et de beaucoup d’intonations.
\subsubsection{groupe B}
\paragraph{}
	Dans ce cas, il s’agit de trois filles se trouvant un peu trop loin pour que je puisse comprendre toute la conversation. L’une d’entre était en train de parler d’un problème qu’elle rencontrait pour se loger pendant que les deux autres écoutaient. La fille qui parler faisaient de grands gestes et parlait fort avec beaucoup d’intonation. Elle était penchée sur la table tandis que les deux autres étaient adossées à leur dossier. 

\subsection{Mixtes}
\paragraph{}
Pour les groupes mixtes, l’observation a eu lieu dans un Fast Food entre 19h30 et 20h30. Je n’ai donc pu observer que des binômes. Dans chacun des cas, je n’ai pas pu entendre de quoi ils parlaient à cause de la distance. Cependant, j’ai pu imaginer le type de conversation grâce au langage corporel.
\subsubsection{couple A}
\paragraph{}
Le couple est assis face à face, surement pour faciliter la communication. L’homme, bien qu’ayant fini de manger, est penché vers la fille au-dessus de la table. Ses mains sont coincées sous ses jambes ce qui est un signe de stress. La fille, quant à elle, est en retrait appuyées sur le dossier de sa chaise. Elle est restée souriante tout au long du repas. L’homme ne peut s’empêcher de la regarder tout le temps même lorsqu’il a débarrassé les plateaux. On peut donc en déduire que ces personnes sont très proches. En effet, l’homme voudrait plus que de l’amitié tandis que la fille ne le considère que comme un très bon ami.  
\subsubsection{couple B}
\paragraph{}
	Le couple ne se situe pas face à face, ils se sont positionnés en diagonale. Ils ne se regardent jamais. La fille utilise ses écouteurs pendant que l’homme regarde son portable. Seul l’homme mange, la fille n’a pas de plateau. Lorsque l’homme part, la fille le regarde dans le dos en levant les yeux au ciel ce qui confirme qu’ils se connaissaient. On peut donc en déduire qu’ils sont en froids et on peut même supposer qu’ils viennent de rompre. Puis un deuxième homme rejoint la fille. A ce moment son attitude change du tout au tout : elle sourit, elle joue avec ses cheveux, elle le regarde sans arrêt. On peut donc en déduire qu’elle est en de bien meilleur termes avec cet homme et on peut même supposer qu’elle souhaite le séduire.
	
\subsection{Conclusion}
\paragraph{}
Suite à ces observations, j’ai pu me rendre compte que les femmes ont une interaction physique entre elles bien plus fréquente qu’entre les hommes : elles se regardent tout au long de la conservation contrairement aux hommes, les hommes se soutiennent tout en restant à distance comme le montre la tape sur l’épaule qui est un geste très bref et assez éloigné du reste du corps. Les hommes parlent souvent de technologies ou de sport tandis que les femmes se plaignent généralement des problèmes qu’elles rencontrent dans leur vie. De plus, les hommes ont tendance à illustrer leurs propos alors que les femmes occupent leurs mains pendant leurs conversations. Enfin, dans les groupes mixtes, l’attitude des gens varie du tout au tout entre les moments où ils veulent séduire leur partenaire et les moments où ceux-ci ne les intéresse pas du point de vue séduction.
\paragraph{}
Comme l’affirme D. Tannen, les femmes s’écoutent parler sans s’interrompre et écoutent les conseils donnés sans problème. De même, les hommes tentent de rester en position de force avec des gestes « viril » comme la tape sur l’épaule.

\section{3\ieme version}
\paragraph{}
\subsection{Hommes}
\paragraph{}
bla

\subsection{Femmes}
\paragraph{}
bla

\subsection{Mixtes}
\paragraph{}
bla

\section{4\ieme version}
\subsection{Hommes}
\paragraph{}
bla

\subsection{Femmes}
\paragraph{}
bla

\subsection{Mixtes}
\paragraph{}
bla

\chapter{Entretiens}
\paragraph{}
bla

\subsection{1\ier version}
\paragraph{}
Dans le cadre de l’exploration « entretien avec un étranger », nous avons discuté avec deux personnes différentes à savoir Carmen Matellan, professeur d’espagnol, et Catherine, stagiaire de l’épi 2.
\paragraph{}
Madame Matellan est arrivée en France il y a 25 ans pour se marier et fuir la dictature. Sa vision de la France a été modifiée après son arrivée. En effet, elle voyait la France comme le pays de la liberté, l’égalité et la fraternité. Cependant, elle s’est rendu compte que ce n’était pas un pays si fraternel et égalitaire que prévu : les études poursuivies sont décidées suite à un concours, toutes les études ne sont pas accessibles à tous à cause de leur coût, la vie est bien plus hiérarchisée qu’en Espagne du temps de la dictature. 
\paragraph{}
Les différences flagrantes entre l’Espagne et la France sont dans les horaires, la nourriture et le pessimisme français opposé à l’optimisme espagnol. Cette différence entraine des difficultés. En effet, elle est souvent prise pour une simplette à cause de son sourire constamment présent. De plus, elle trouve que les français sont un peu hypocrite dans leur politesse et dans leur manque de solidarité : ils se permettent tout et n’importe quoi à partir du moment où ils s’excusent notamment dans le métro.
\paragraph{}
Elle est émerveillée tous les jours par la beauté de Paris et la culture française telle que l’histoire, la littérature ou la beauté de la langue elle-même. C’est pourquoi elle est choquée par le faible niveau de français des jeunes. En effet, beaucoup de jeunes font de nombreuses fautes de français et détruise ainsi la langue.
\paragraph{}
Catherine est une étudiante arrivée en Août dernier pour faire un stage d’un an. Elle s’occupe de créer les listes de présence des groupes de langue. Elle vient de la banlieue Londonienne. Elle a décidée de venir en France pour améliorer son français. En effet, elle étudie la philosophie et le français à l’université de Liverpool.
\paragraph{}
Avant de partir, elle pensait que Paris était plus propre et plus grand. Elle pensait aussi se faire plus d’amis français avec qui discuter.  Elle trouve que les gens sont mieux habillés en France et que la vie est plus chère. Enfin,  elle a été surprise plusieurs fois par le fait que des hommes l’ont suivi dans la rue ce qui lui a un peu fait peur.


\subsection{2\ieme version}
\paragraph{}
Takamichi Ishii est un étudiant japonais de 22 ans arrivé en France en février. Il a étudié deux mois à l’ESIEE avec des compatriotes et a accepté de répondre à nos questions. À sa demande, nous l’appelions Taka - il nous a confié être appelé différemment par chacun de ses camarades. Si le niveau de familiarité en France se distingue par l’usage du prénom et du tutoiement, on trouve en japonais de nombreuses façons d’appeler quelqu’un, en utilisant des suffixes comme -kun ou -san ; le prénom seul n’est utilisé que par les proches.
\paragraph{}
Taka est parti de Kanagawa (près de Tokyo) pour un voyage d’études et a choisi la France pour sa renommée touristique. Il nous a confié apprécier notre pays et vouloir y rester plus longtemps. Nous l’avons interrogé sur la prétendue impolitesse des Français, dont on parle beaucoup dans les médias, mais Taka nous trouvait au contraire plutôt agréables. Second cliché évincé : selon lui, les Français parleraient selon lui un bon Anglais, bien meilleur en tout cas  que celui des Japonais ; ce qui était un avantage puisque la langue française lui paraissait très complexe. Nous serions surtout moins timides qu’eux, ce qu’il considérait comme une qualité (nous en étions moins sûrs).
\paragraph{}
Un point négatif tout de même : la région parisienne lui a paru sale, du RER aux bords de routes en passant par les sanitaires. Il faut dire que, selon Taka, les Japonais sont plus surveillés. Lorsque nous lui avons demandé de se rappeler un évènement qui l’avait choqué, il a tout de suite évoqué les fraudeurs du RER. Il est impossible au Japon de sauter par-dessus les portiques, puisque les gardiens n’hésitent pas à intervenir ; il était déconcerté de voir les employés de la RATP rester passifs, désabusés.
\paragraph{}
Nous en sommes venus à parler du choc des cultures : Taka estime que tout est différent entre les quotidiens européen et japonais. On vient de l’évoquer : si nous nous plaignons de l’augmentation de la surveillance, elle reste bien moins importante qu’au Japon. Une conséquence, sans doute : les pickpockets sont beaucoup plus répandus chez nous, ce qui l’a surpris : l’un de ses amis s’était déjà fait volé son téléphone. Les trains sont bien plus souvent en retard. Taka nous a également parlé de différences plus triviales, mais qui lui avaient sauté aux yeux : la rareté des distributeurs automatiques, qui au Japon sont parlants, vendent de tout et surgissent tous les deux mètres ; les Français peuvent garer leur voiture sur le côté de la rue et pas uniquement dans des parkings ; les rues portent des noms de célébrités, et pas seulement des numéros.
\paragraph{}
En somme, ce qui frappe le plus les Japonais en France, ce sont les détails ; ceux qui font que malgré une grande occidentalisation de la culture japonaise, le quotidien reste toutefois très différent.



\chapter{Dislocation}
\paragraph{}
bla

\subsection{1\ier version}
\paragraph{}
bla

\subsection{2\ieme version}
\paragraph{}
bla

\subsection{3\ieme version}
\paragraph{}
bla

\subsection{4\ieme version}
\paragraph{}
bla

\chapter{Enfants}
\paragraph{}
bla

\subsection{1\ier version}
\paragraph{}
bla

\subsection{2\ieme version}
\paragraph{}
bla

\subsection{3\ieme version}
\paragraph{}
bla

\subsection{4\ieme version}
\paragraph{}
bla

\part{A chacun sa culture qui fait rêver}
\chapter{Le Chili}

\chapter{Le Japon à travers les animes et les mangas}

\chapter{Le Japon : Matsuris et Croyances}
\section{Les croyances ...}
\paragraph{}
Le Japon ne possède pas de religion particulière. En effet, les japonais auront tendance à se considérer comme appartenant à aucune religion ou à plusieurs religions en même temps. Les religions principales sont le shintoïsme et le bouddhisme mais il y a une minorité de chrétiens et de musulmans. 
\paragraph{}
Ce mélange de religion est possible grâce aux caractéristiques des religions principales du pays. Le shintoïsme est une religion polythéiste tandis que le bouddhisme correspond plus à une philosophie de vie accompagnée de méditation.
\paragraph{}
« On dit souvent que le Japonais naît, grandit et s’amuse shinto, s’éduque confucéen, se marie chrétien, vit dans l’irréligion et meurt bouddhiste ». (Le Japon des Japonais par Philippe PONS et Pierre-François SOUYRI)

\section{... à l'origine des matsuris}
\paragraph{}
Le shintoïsme et le bouddhisme sont à l’origine de nombreuses fêtes traditionnelles autrement appelées Matsuri. En effet, pour attirer la bienveillance des dieux, chacun d’entre eux est prié et vénéré lors de rituel et de fête. Bien que les japonais ne croient pas vraiment en ces dieux, ils accomplissent ces rituels au cas où les dieux existeraient et retireraient leur bienveillance envers les japonais.
\paragraph{}
Ces fêtes ont lieu régulièrement tout au long de l’année. Au printemps, les Japonais fêtent le repiquage du riz et prient pour se protéger des épidémies. En été, ils prient pour se protéger contre les typhons et les ravages causés par les insectes ainsi que pour leurs ancêtres.  En hiver, ils prient pour la nouvelle année. Il y a aussi des fêtes locales pour prier les dieux locaux à différents moments de l’année. 
\paragraph{}
Lors de ces festivités, les hommes défilent dans le quartier en portant le mikoshi (sorte d’autel portatif) sur leurs dos. Ils sont vêtus d’un happi (veste découvrant leurs torses) et d’un fundoshi (sorte de cache-sexe traditionnel). Une fois l’autel retourné au temple, la fête continu le long de la rue du temple où l’on peut trouver des stands de friandises en sucres, de nouilles sautées, de porte-bonheurs… 
\begin{center}
\includegraphics[scale=0.07]{mikoshi.jpg}
\includegraphics[scale=1.3]{odori.jpg}
\end{center}
\paragraph{}
Dans les plus grands matsuris, de grands chars défilent à travers le quartier accompagnés des flûtes, tambours et gongs. Ces chars sont aussi accompagnés de troupes de danseurs. Les filles profitent de ces festivals pour sortir leurs kimonos pour rentrer dans l’ambiance de la fête.
\begin{center}
\includegraphics[scale=1.3]{gion.jpg}
\end{center}
\newpage
\section{Les matsuris les plus connus}
\begin{itemize}
\item Aoi matsuri, 15 mai, Kyoto
\item Aomori nebuta matsuri, 2-7 août, Aomori
\item Awa-Odori, 12-15 août, Tokushima
\item Danjiri matsuri, deuxième week end de septembre, Kishiwada
\item Etchu owara kaze no bon, 1-3 septembre Toyama
\item Gion matsuri, juillet, Kyoto quartier de Gion
\item Gozan no Okuribi, 16 août, Kyoto
\item Hadaka matsuri, troisième samedi de février, Okayama
\item Hakata Gion Yamakasa, 1-15 juillet, Hakata
\item Ise-jingu kannamesai et ninamesai, 15-25 octobre et 23-29 novembre, Ise
\item Jidai matsuri, 22 octobre, Kyoto
\item Kanda Matsuri, deuxième dimanche de mai, Tokyo
\item Namahage, 31 décembre, Oga
\item Narita-san setsubun-e, 3 février, Narita
\item Onbashira, avril tous les six ans, Suwa
\item Otaue matsuri, 14 juin, Osaka
\item Sanja matsuri, troisième week end de mai, Tokyo
\item Sanno matsuri, 14-15 avril, Takayama
\item Sendai Tanabata matsuri, 6-8 août, Sendai
\item Sentei-sai, 3-4 mai, Shimonoseki
\item Festival de la neige de Sapporo, mi-février, Sapporo
\item Tenjin matsuri, 24-25 juillet, Osaka
\item Yosakoi matsuri, 9-12 août, Kochi
\end{itemize}

\chapter{Les elfes}
\part{Les Otakus ou les passionnés de mangas.}
\chapter{Qu'est-ce qu'un Otaku ?}
\paragraph{}
Le terme Otaku est composé du préfixe honorifique "o" et du mot "taku" qui peut se traduire par "maison". Il a plusieurs significations. En effet, au Japon, il désigne une personne cloitrée chez elle et s'adonnant à une passion pouvant se développer à l'intérieur. Au contraire, en Europe, le terme Otaku désigne une personne passionnée par les mangas et les animes. En France, les Otakus ont développer une culture qui leur est propre.
\paragraph{}
Voici la définition d'une culture d'après le Larousse:
"Dans un groupe social, des personnes appartennant à la même culture partagent un ensemble de signes caractéristiques qui les différencies des personnes n'appartennant pas à cette culture. En effet, ces personnes partagent un vocabulaire, un style vestimentaire et un centre d'intéret commun." (larousse.fr)
\paragraph{}
Les Otakus, quant à eux, partagent leur intéret pour les mangas et les animes. Cet intéret s'exprime à travers leur lecture et leur visionnage de série bien évidemment. Mais il s'exprime aussi à travers leur vocabulaire emprunté à la langue japonaise et leur cosplay pour les plus passionnés. 

\chapter{L'histoire du manga: de sa création à son évolution en France}

\section{Les origines du manga}
\paragraph{}
Le terme « manga » a été inventé en 1814 par Katsushika Hokusai lors de la publication de son recueil de dessin représentant la vie du peuple japonais durant l’ère Edo intitulé « \underline{Hokusai Manga} ». Ce terme est issu de deux idéogrammes chinois signifiant littéralement « image dérisoire ». 
\paragraph{}
Cependant, les premiers mangas existaient déjà au XIIème  siècle. C’était des rouleaux de dessins appelés e-makimono qui représentaient des animaux anthropomorphes reproduisant des scènes de la vie de l’époque. Il s’agit du premier manga humoristique.
\begin{center}
\includegraphics[scale=0.7]{emakimono.jpg}
\end{center}
\paragraph{}
Au XIXème siècle, la Japon subit une vague d’occidentalisation tandis que les journaux satiriques se développent de plus en plus en occident. C’est à cette époque que le manga s’industrialise. Au XXème siècle, des séries apparaissent dans les journaux quotidiens et de nouveaux magazines pour enfants consacrés aux mangas sont créés tels que les périodiques \underline{Shônen Club} en 1914 et \underline{Shôjo Club} en 1923.

\section{La naissance du manga moderne}
\paragraph{}
Au lendemain de la Seconde Guerre Mondiale, les mangas se sont multipliés pour répondre à une forte demande de distraction bon marché. De plus, les comics arrivant au Japon inspirent les auteurs japonais. 
\paragraph{}
L’un des auteurs les plus marquants de cette époque fut Osamu Tezuka. Il publie son premier manga en 1946. Il rêve de se lancer dans le dessin animé mais n’en a pas les moyens. Pour pouvoir animer ses histoires, il créer un style graphique et de mise en page dans le but de faire ressentir les mêmes émotions que la vision d’un film : il dessine des cases de tailles et de forme variable pour donner une dynamique cinématographique, il dessine la même action sous de nombreux angles différents pour donner un impression de ralenti, il ajoute des traits de vitesse et enfin il fait de grands yeux très expressifs à ses personnages pour accentuer leurs sentiments et leurs émotions. Cette technique fut ensuite utilisée par tous les autres auteurs jusqu’à aujourd’hui. 
\begin{center}
\includegraphics[scale=0.4]{astroboy.jpg}
\end{center}
\paragraph{}
En 1962, Tezuka crée le Studio Mushi où il adapte ses succès en anime. Il installe ainsi le système impliquant qu’un manga à succès est généralement adapté en anime.

\section{L'introduction du manga en France}
\paragraph{}
A la fin des années 70, la production européenne de dessin anime est insuffisante pour répondre à la demande. Les grandes chaines se tournent alors vers les productions japonaises qui sont nombreuses, variés et peu chère. 
\paragraph{}
Ainsi, en 1978, Antenne 2 diffuse \underline{Goldorak} qui eut un succès inattendu considérable. En effet, la série eut même droit de faire la couverture de Paris Match intitulée « la folie Goldorak ».  Suite à ce succès, Antenne acquière et diffuse trois autres série : \underline{Albator}, \underline{Capitaine Flam} et \underline{Candy}.
\begin{center}
\includegraphics[scale=0.4]{goldorak.jpg}
\end{center}
\paragraph{}
En 1983, Antenne 2 diffuse \underline{Les Mystérieuses Cités D’OR} qui est une réalisation conjointe des studios français, luxembourgeois et japonais. 
\paragraph{}
Face au succès des animes japonais sur Antenne 2, La Cinq commence a diffusé ses propres animes japonais tels \underline{Jeanne et Serge} en 1987 et \underline{Olive et Tom} en 1988. Enfin, TF1 crée l’émission le Club Dorothée dans laquelle elle diffuse les plus gros succès du moment au Japon : \underline{Saint Seiya}, \underline{Ken le Survivant} et \underline{Dragon Ball}. 

\section{La polémique \underline{Ken le Survivant}}
\paragraph{}
Cependant ces animes venu du Japon font polémique. En effet, en 1989, Ségolène Royal publie \underline{Le Ras-le-bol des bébés zappeurs} dans lequel elle reproche aux animes japonais de montrer une trop grande violence. Ses propos illustrent les préjugés forgés à l’époque. 
\begin{center}
\includegraphics[scale=0.15]{ken.jpg}
\end{center}
\paragraph{}
L’anime le plus critiqué est \underline{Ken le Survivant}. En effet, cet anime diffusé parmi les dessins animés pour enfants raconte l’histoire de Ken qui erre dans un monde post-apocalyptique en aidant les innocents à se protéger face aux crapules grâce aux arts martiaux. De plus, les doubleurs français ont truffé la version française de jeux de mots plus ou moins intelligent.

\section{Des années 90 à aujourd'hui}
\paragraph{}
En 1993, le Club Dorothée diffuse \underline{Dragon Ball Z} qui rencontre un  succès tel qu’il sera ensuite diffusé sous format papier. De nos jours, les animes sur les chaines généralistes ne sont que les rediffusions des grands succès du début des animes japonais en France.  En effet, pour trouver des animes plus récent, il faut passer par la TNT, les chaines spécialisés ou les chaines payantes. 

\section{Le début du manga papier}
\paragraph{}
Entre 1978 et 1982, un magazine intitulé \underline{Le Cri Qui Tue} essaye sans succès de publier des mangas. En 1983, la maison d’édition Les Humanoïdes Associés tentent de publier \underline{Gen d’Hiroshima} mais le public n’est pas au rendez-vous. 
\paragraph{}
En 1988, Jacques Glénat rentre du Japon avec un manga intitulé \underline{Akira} qu’il juge très prometteur. Il apparait d’abord en 1990 sous forme de fascicule couleurs mais ne rencontre pas le succès. En 1991, la sortie du film tiré de ce manga va éveiller un vif intérêt des français envers l’art des mangakas. Glénat publie ensuite \underline{Akira} dans un format livre de poche ayant ainsi un grand succès. 
\begin{center}
\includegraphics[scale=0.3]{dragonball.jpg}
\end{center}
\paragraph{}
Mais les mangas en version papier connaissent un véritable succès seulement avec la sortie de \underline{Dragon Ball} en 1993. Suite à ce succès, en 1995, Casterman lance sa collection manga et Dargaud crée sa filiale Kana.  Il existe 	aujourd’hui trente-trois éditeur spécialisé en France.

\section{Les films d'animation}
\paragraph{}
Le plus grand honneur pour un mangaka c’est de voir son manga adapté en anime puis en film d’animation. Le premier film d’animation tiré d’un manga qui est sorti en France fut \underline{Akira} en 1991. Puis le film \underline{Ghost In The Shell} de Mamoru Oshii sort en 1997. Sa suite, \underline{Innocence}, est sélectionnée au festival de Cannes en 2004.
\paragraph{}
En 1995, \underline{Porco Rosso} de Hayao Miyazaki sort au cinéma en France. Mais c’est en 2000 que Miyazaki conquis le public français avec \underline{Princesse Mononoké}. Suite au triomphe du \underline{Voyage de Chihiro}, les anciens films de Miyazaki sont diffusés en France. 
\begin{center}
\includegraphics[scale=0.2]{miyazaki.jpg}
\end{center}
\paragraph{}
En 2003, sort \underline{Interstella 5555} qui est une mise en scène de l’album « Discovery » de Daft Punk par Leiji Matsumoto le créateur d’\underline{Albator}.

\section{Les raisons d'un tel succès}
\paragraph{}
A l’origine, les œuvres japonaises ont été introduites en France pour satisfaire un public d’enfants. Cependant, elles ont très vite intéressées les adolescents. 
\paragraph{}
Cet engouement peut s’expliquer par un tout nouveau type de réalisation : la mise en scène est dynamique et cinématographique. En effet, il y a une alternance continue entre les différents types de cadrages tels que des gros plan sur les visages, les plans larges pour découvrir le décor, des mouvements de caméra et la technique du face à face entre deux personnages présentant sur le même plan le dos d’un personnage et la face d’un deuxième personnage. 
\paragraph{}
De plus, les scénarios innovants rendent les mangas très attrayant. En effet, les séries se suivent comme des feuilletons ce qui permet de développer des scénarios complexes. Les personnages ont des psychologies très développées et qui évolue tout au long de la série suite aux expériences qu’ils traversent. Enfin, il arrive qu’il y ait des flashbacks lorsqu’un personnage se remémore un évènement ce qui permet aux spectateurs ou lecteurs de comprendre l’histoire même s’ils ont loupés un épisode. 
\paragraph{}
Les mangas sont également adaptés au public visé : il existe des Shônen pour les jeunes garçons qui mettent en scène des affrontements entre les personnages, des Shôjo pour les jeunes filles qui regroupent des histoires de romances, de jeunes filles aux pouvoirs magiques et de jeunes sportives. Ce genre eut un véritable succès grâce à un manque de concurrence au sein des bandes dessinées belges. En effet, les thèmes abordés sont nouveaux, les personnages sont souvent des adolescents et le rythme de publication soutenu permet au personnage de vieillir avec le lecteur ce qui permet au lecteur de s’identifier au personnage. Il existe d’autres genres tels que les Seinen ou les Josei destiné aux jeunes adultes et dont les thèmes sont plus matures et la violence plus crue.
\begin{center}
\includegraphics[scale=0.8]{shojo-shonen.jpg}
\end{center}
\paragraph{}
L’aspect de chaque personnage est étudié dans les moindres détails : les yeux sont toujours assez grands pour être très expressifs et les chevelures sont d’un style séduisant mais différent pour permettre un attachement à celui-ci. De plus, les tenues sont plus ou moins réfléchies suivant le contexte de l’histoire : dans un lycée, les personnages porteront un uniforme tandis que dans une série historique ou fantastique, les tenues seront réfléchies au détail près. 
\paragraph{}
Enfin, l’impression en noir et blanc et le format poche donne un style épuré et un coût faible à l’œuvre. 

\section{Des fans impliqués}
\paragraph{}
Dans les années 90 apparait des fanzines autrement dit des magazines créés par des fans pour des fans. En effet, en 1991 a lieu la première parution d’\underline{AnimeLand}. Il est encore en activité aujourd’hui. En 2006, les Humanoïdes Associés crée le \underline{Shogun Mag}, un magazine de prépublication comme il en existe au Japon. 
\begin{center}
\includegraphics[scale=0.8]{animeland.jpg}
\end{center}
\paragraph{}
Internet a aussi permis aux mangas de se propager. En effet, les fans ont créés les fansubs et les scantrads. Le fansub consiste à se regrouper en « team » pour traduire et sous titrer les animes. Le scantrad, quant-à-lui, consiste à scanner des mangas papier pour ensuite les traduire. Bien qu’étant illégal, cette pratique concerne principalement des séries non licenciés en France ce qui permet de promouvoir et d’évaluer la popularité potentielle de celles-ci.

\section{Un Japon qui fascine et qui inspire}
\paragraph{}
Les mangas permettent aussi de découvrir la culture japonaise. En effet, les mangas ont tendances à décrire certains aspects de la culture japonaise. 
\paragraph{}
En 1999, la Japan Expo est créée. Il s’agit d’un festival regroupant des stands sur les mangas, les animes, les jeux vidéo, le cinéma, la J-Music et la culture japonaise. Il permet aussi aux fans de faire du cosplay, une activité qui consiste à se déguiser comme son personnage favori du moment. 
\begin{center}
\includegraphics[scale=0.3, angle=-90]{erza.jpg}
\end{center}
\paragraph{}
L’art du manga fascine aussi les auteurs de bande dessinée français. En effet, en 2005, Jean Giraud collabore avec Jirô Taniguchi pour créer \underline{Icare}. De plus, on assiste à l’apparition de « manfra », mangas français, tels que \underline{Dofus} ou \underline{DreamLand}.

\chapter{Un vocabulaire}

\section{Catégories de manga en fonction du public visé}

\begin{center}
	\includegraphics[scale=0.5]{Kodomo.jpg}
\end{center}

\begin{description}
	\item[Kodomo:] Anime ou manga destiné aux jeunes enfants.
	\item[Shonen:] Anime ou manga destiné aux jeunes garçons (spécifiquement
		pour des personnes de moins de 14 ans mais qui peut avoir une audience
		beaucoup plus large).
	\item[Shojo:] Anime ou manga destiné aux jeunes filles.
	\item[Seinen:] Anime ou manga destiné aux hommes qui sont dans les environs
		de la 50aine.
	\item[Josei:] Anime ou manga destiné aux femmes allant de la fin de
		l'adolescence à la femme adulte.
	\item[Ecchi:] Anime, manga ou jeu érotique.
	\item[Hentai:] Anime, manga ou jeu pornographique.
\end{description}

\section{Catégorie de manga en fonction de l'univers}

\begin{center}
	\includegraphics[scale=0.8]{Madoka.jpg}
\end{center}

\begin{description}
	\item[Magical girl (Mah\=o Sh\=ojo):] Manga ou anime mettant en scène de
		jeunes filles ayant des pouvoir magiques, généralement de type sorcière
		ou magicienne (Par exemple: Madoka$\star$Magika, Sailor Moon).
	\item[Harem:] Manga ou anime ayant un personnage masculin proéminent pour
		un nombre plus important de personnages féminins principaux.
	\item[Mecha:] Manga ou anime mettant en scène des ``mecha'', à savoir des
		robots humanoïdes de taille gigantesque (Par exemple: Neon Genesis
		Evangelion, Goldorak)
\end{description}

\section{Suffixes honorifique}

\begin{description}
	\item[San:] Niveau de politesse standard (équivalent à Mr./Mme.)
	\item[Kun:] Très souvent utilisé pour s'adresser à des garçons ou des amis
		masculins.
	\item[Chan:] Utilisé pour les bébés, jeunes filles ou jeunes garçons, ou
		pour les petits-amis/amis très proches.
	\item[Sensei:] Utilisé pour les professeurs, politiciens, docteurs ou
		autres personnes d'autorité.
	\item[Sama:] Utilisé pour les divinités et les personnes de la royauté.
	\item[Senpai:] Utilisé pour les supérieurs ou les personnes respectées.
\end{description}

\section{Caractère type de personnages}

\begin{center}
	\includegraphics[scale=0.5]{Yandere.jpg}
\end{center}

\begin{description}
	\item[Yandere:] Personnage psychologiquement instable ayant des
		sentiments pour un autre personnage mais utilisant ses pulsions
		meurtrières pour par exemple ``se débarrasser de la compétition''.
	\item[Tsundere:] Personnage dur et énervant au premier abord mais
		affectueux un fois sortie de sa coquille.
	\item[Kuudere:] Personnage froid et cynique en apparence mais attentionné
		en réalité.
	\item[Deredere:] Personnage adorable et énergique envers tout le monde.
	\item[Dandere:] Personnage calme et silencieux lorsque entouré par
		plusieurs personnes mais adorable et énergique lorsque seul avec une
		autre personne.
	\item[Kamidere, Himedere, Oujidere:] Personnage ayant un complexe
		de supériorité et se croient respectivement l'équivalent d'un
		Dieu (Kami), Princesse (Hime) ou Prince (\=Oji).
\end{description}

\section{Style de dessins}

\begin{center}
	\includegraphics[scale=0.3]{Moe.png}
\end{center}

\begin{description}
	\item[Moe:] Style de dessin caractéristique et simpliste qui a pour but de
		faire sentir des sentiments d'affection envers les personnages (Par
		exemple: Nichijou).
\end{description}

\section{Idéalisation}

\begin{center}
	\includegraphics[scale=0.15]{Bishounen.jpeg}
\end{center}

\begin{description}
	\item[Bishonen:] Représentation de l'idéal masculin, en majorité un garçon
		efféminé, androgyne et jeune.
	\item[Bishojo:] Représentation de l'idéal féminin.
\end{description}

\chapter{Les produits dérivés}

\chapter{Les conventions}
\section{L'origine des conventions}
\paragraph{}
Les rencontres dédiées aux mangas et anime sont anciennes au Japon. Une pionnière du genre est Comiket, créée en 1975 à Tokyo dans le but de populariser des d\=ojinshi, mangas créés et publiés par des fans. Elles se multiplient au fil des années avec l’aide des éditeurs, qui y voient un bon moyen d’assurer la publicité de leurs séries.
\paragraph{}
Le phénomène se développe également en Occident dans les années 1980 avec l’importation de la culture japonaise. Aux États-Unis, ces anime conventions rejoignent les festivals de culture populaire américaine : science-fiction, comic books, etc.. Leur véritable succès ne commencera toutefois que dans les années 1990, décennie durant laquelle de tels évènements apparaissent aussi en Europe. En France, la plus célèbre est Japan Expo, créée en 1999, qui lance pour de bon le phénomène des conventions françaises (l’anglicisme convention ayant été adopté pour les différencier des expositions plus classiques). D’autres évènements suivent la tendance : Paris Manga, Japan Expo Sud… Leur point commun est que leur croissance rapide les a poussés à s’ouvrir à une culture geek plus générale, invitant des créateurs de séries, jeux vidéo, films de SF, etc.
\begin{center}
\includegraphics[scale=0.3]{japanexpo.jpg}
\end{center}

\section{Le contenu des conventions}
\paragraph{}
Les invités d’honneur sont au centre de ces évènements : il peut s’agir d’un mangaka, d’un scénariste ou d’un designer célèbre présent pour des dédicaces ou une conférence. Ils peuvent également tenir un atelier. De nombreux autres artistes tiennent des stands de dédicace. Ceux-ci sont parfois tenus par un éditeur occidental, qui compte sur la présence des artistes pour faire découvrir une série en vue d’une publication.
\paragraph{}
On y trouve aussi de nombreux stands de vente, où sont proposés des mangas mais également de nombreux produits dérivés : peluches, posters, et même épées. Il est rare de revenir d’une convention sans un ou deux goodies.
\begin{center}
\includegraphics[scale=0.5]{boutique.png}
\end{center}
\paragraph{}
Les salles de conférence sont également employées pour des avant-premières et projections tests, souvent suivies de panels où des artistes répondent aux questions des fans. Des concours y ont également lieu, notamment de cosplay, discipline chère aux conventions. Les fans y trouvent un moyen d’exprimer leur créativité en confectionnant des tenues inspirées ou directement reprises de celles de personnages de mangas ou jeux vidéo. Le phénomène est sans doute celui dont la croissance a été la plus sensible ces dernières années, parfois au regard moqueur des médias généralistes.
\begin{center}
\includegraphics[scale=0.3]{cosplay.jpg}
\end{center}
\paragraph{}
Enfin, les conventions proposent des expositions : d’art japonais, de dessins préparatoires, etc.
\begin{center}
\includegraphics[scale=0.4]{demo.png}
\end{center}
\chapter{Interview de deux Otakus}
\paragraph{}
Nous avons interviewé deux personnes passionés de mangas ou animes dont nous ne divilgueront pas les noms. Nous les appellerons donc A et B.\\
\\
\textbf{Depuis quand lis-tu des mangas/regardes tu des animes?}\newline
A: Je regarde des animes depuis environ 1 ans et demi. Je n'ai jamais lu de mangas.
	\newline
B: Je lis des mangas depuis l'âge de 9 ans mais j'ai commencé à regarder des animes à l'âge de 6 ans.
	\newline
\textbf{Quel fut ton premier manga/anime?}
\newline
A: Le premier anime que j'ai regardé été \underline{Naruto}.
\newline
B: Le premier anime que j'ai vu été \underline{Olive et Tom} à la télé et le premier manga que j'ai lu été \underline{Dragon Ball}.
\newline
\textbf{Quel type d'anime/manga lis'tu en général?}
\newline
A: Je regarde tous les genres: shojo, shonen, seinen...
\newline
B:Je regarde et lis beaucoup de genres différents: de la science fiction, du sport, de la fantaisy...
\newline
\textbf{Qu'est-ce qui t'attire dans les mangas/animes?}
\newline
A: C'est la folie du genre qui m'as attiré dans les animes.
\newline
B: Ce qui m'a attiré dans les mangas et les animes c'est leur capacité de rendre génial un sujet stupide.
\newline
\textbf{Quel est ton manga/anime favori?}
\newline
A: Je dirai que mon anime préféré est \underline{FullMetal Alchemist Brotherhood}.
\newline
B: Je ne pourrai pas citer un anime/manga en particulier: il y en a trop de bien.
\newline
\textbf{Préfères-tu les mangas ou les animes? Pourquoi?}
\newline
A: Je ne lis jamais de mangas, je ne regarde que des animes.
\newline
B: Je préfère les mangas, je trouve que les animes commencent à manquer d'originalité: ils se ressemblent de plus en plus.
\newline
\textbf{Préfères-tu la vo ou la vf pour les animes?}
\newline
A: Ben la vo.
\newline
B: La vo.
\newline
\textbf{As-tu vu des films d'animation japonais? Combien? De quel réalisateur?}
\newline
A: Au moins une dizaine du stucio Ghibli.
\newline
B: Oula plein.
\newline
\textbf{Es-tu déjà allé à une convention? Laquelle ? Combien de fois ? Qu'est-ce qui t'as plu?}
\newline
A: Non jamais.
\newline
B: J'y vais à chaque fois que je peux. J'ai souvent fait la Japan Expo et parfois Paris Manga. Mais comme ça coute cher je n'ai fait que des conventions sur Paris.
\newline
\textbf{As-tu déjà fait du cosplay?}
\newline
A: Oui, une fois en Pikachu.
\newline
B: Jamais.
\newline
\textbf{Lis-tu des scans ? Regardes-tu en streaming ou téléchargement?}
\newline
A: Je ne regarde que en streaming ou téléchargement.
\newline
B: Je commence souvent mes séries en scan pour les découvrir mais parfois je les achète car je trouve que certaine série mérite qu'on possède tous les tomes chez soi.

\part{Concepts théoriques}
\chapter{Monochronique et Polychronique}

\chapter{Communication haut contexte et bas contexte}

\chapter{4 dimensions culturelles}

\chapter{Sphères publiques et privées}
\part{Des sources utiles}

\chapter{Blogs de français à l'étranger}
\section{blog 1:}


\chapter{Blogs d'étranger en France}
\section{nlog 1:}

\chapter{Sites sur l'interculturalité}
\section{site 1:}


\chapter{Nos sources pour ce dossier}
\section{A chacun sa culture qui fait rêver}
\section{Les Otakus ou les passionnés de mangas}
\subsection{Histoire du manga}
\noindent
\url{http://www.france-jeunes.net/lire-le-manga-et-la-france-analyse-d-un-succes-25649.htm}\\
\url{http://hitek.fr/actualite/dossier-manga-naissance-arrivee-france_1924}

\subsection{Vocabulaire}
\noindent
\url{https://en.wikipedia.org/wiki/Glossary_of_anime_and_manga}\\
\url{https://www.asianfanfics.com/blog/view/411262/url}

\section{Concepts théoriques}
\subsection{4 Dimensions culturelles}
\noindent
\url{http://geerthofstede.nl/dimensions-of-national-cultures}\\
\url{http://news.telelangue.com/en/2011/09/cultural-theory}

\part{Présentation des membres}
\section{Jordan Singkouson}
\paragraph{}
Jordan Singkouson, étudiant d’origine laotienne et française habitant la ville de Bussy-Saint-Georges, un paisible endroit situé au cœur de la seine-et-marne. 

\section{Elodie Caroy}
\paragraph{}
Elodie CAROY, 19 ans, étudiante en bac+2 dans une école d’ingénieur, aime le dessin et a visité quelques pays d’Europe.

\section{Rémi Nicole}
\paragraph{}
Rémi NICOLE, 18 ans, étudiant en seconde année de préparation intégrée à l’ESIEE Paris, mélomane et fan de la culture japonaise.

\section{Timothé Pallot}
\paragraph{}
Timothée PALLOT, 19 ans, étudiant en 2e année à ESIEE Paris, amateur de cinéma, création audiovisuelle et de jeux vidéo.


\end{document}
% vim: spell : spelllang=fr
