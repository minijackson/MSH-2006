\part{Les Otakus ou les passionés de mangas.}

\chapter{L'histoire du manga: de sa création à son évolution en France}

\section{Les origines du manga}
\paragraph{}
Le terme « manga » a été inventé en 1814 par Katsushika Hokusai lors de la publication de son recueil de dessin représentant la vie du peuple japonais durant l’ère Edo intitulé « Hokusai Manga ». Ce terme est issu de deux idéogrammes chinois signifiant littéralement « image dérisoire ». 
\paragraph{}
Cependant, les premiers mangas existaient déjà au XIIème  siècle. C’était des rouleaux de dessins appelés e-makimono qui représentaient des animaux anthropomorphes reproduisant des scènes de la vie de l’époque. Il s’agit du premier manga humoristique.
\paragraph{}
Au XIXème siècle, la Japon subit une vague d’occidentalisation tandis que les journaux satiriques se développent de plus en plus en occident. C’est à cette époque que le manga s’industrialise. Au XXème siècle, des séries apparaissent dans les journaux quotidiens et de nouveaux magazines pour enfants consacrés aux mangas sont créés tels que les périodiques Shônen Club en 1914 et Shôjo Club en 1923.

\section{La naissance du manga moderne}
\paragraph{}
Au lendemain de la Seconde Guerre Mondiale, les mangas se sont multipliés pour répondre à une forte demande de distraction bon marché. De plus, les comics arrivant au Japon inspirent les auteurs japonais. 
\paragraph{}
L’un des auteurs les plus marquants de cette époque fut Osamu Tezuka. Il publie son premier manga en 1946. Il rêve de se lancer dans le dessin animé mais n’en a pas les moyens. Pour pouvoir animer ses histoires, il créer un style graphique et de mise en page dans le but de faire ressentir les mêmes émotions que la vision d’un film : il dessine des cases de tailles et de forme variable pour donner une dynamique cinématographique, il dessine la même action sous de nombreux angles différents pour donner un impression de ralenti, il ajoute des traits de vitesse et enfin il fait de grands yeux très expressifs à ses personnages pour accentuer leurs sentiments et leurs émotions. Cette technique fut ensuite utilisée par tous les autres auteurs jusqu’à aujourd’hui. 
\paragraph{}
En 1962, Tezuka crée le Studio Mushi où il adapte ses succès en anime. Il installe ainsi le système impliquant qu’un manga à succès est généralement adapté en anime.

\section{L'introduction du manga en France}
\paragraph{}
A la fin des années 70, la production européenne de dessin anime est insuffisante pour répondre à la demande. Les grandes chaines se tournent alors vers les productions japonaises qui sont nombreuses, variés et peu chère. 
\paragraph{}
Ainsi, en 1978, Antenne 2 diffuse Goldorak qui eut un succès inattendu considérable. En effet, la série eut même droit de faire la couverture de Paris Match intitulée « la folie Goldorak ».  Suite à ce succès, Antenne acquière et diffuse trois autres série : Albator, Capitaine Flam et Candy.
\paragraph{}
En 1983, Antenne 2 diffuse Les Mystérieuses Cités D’OR qui est une réalisation conjointe des studios français, luxembourgeois et japonais. 
\paragraph{}
Face au succès des animes japonais sur Antenne 2, La Cinq commence a diffusé ses propres animes japonais tels Jeanne et Serge en 1987 et Olive et Tom en 1988. Enfin, TF1 crée l’émission le Club Dorothée dans laquelle elle diffuse les plus gros succès du moment au Japon : Saint Seiya, Ken le Survivant et Dragon Ball. 

\section{La polémique Ken le Survivant}
\paragraph{}
Cependant ces animes venu du Japon font polémique. En effet, en 1989, Ségolène Royal publie Le Ras-le-bol des bébés zappeurs dans lequel elle reproche aux animes japonais de montrer une trop grande violence. Ses propos illustrent les préjugés forgés à l’époque. 
\paragraph{}
L’anime le plus critiqué est Ken le Survivant. En effet, cet anime diffusé parmi les dessins animés pour enfants raconte l’histoire de Ken qui erre dans un monde post-apocalyptique en aidant les innocents à se protéger face aux crapules grâce aux arts martiaux. De plus, les doubleurs français ont truffé la version française de jeux de mots plus ou moins intelligent.

\section{Des années 90 à aujourd'hui}
\paragraph{}
En 1993, le Club Dorothée diffuse Dragon Ball Z qui rencontre un  succès tel qu’il sera ensuite diffusé sous format papier. De nos jours, les animes sur les chaines généralistes ne sont que les rediffusions des grands succès du début des animes japonais en France.  En effet, pour trouver des animes plus récent, il faut passer par la TNT, les chaines spécialisés ou les chaines payantes. 

\section{Le début du manga papier}
\paragraph{}
Entre 1978 et 1982, un magazine intitulé Le Cri Qui Tue essaye sans succès de publier des mangas. En 1983, la maison d’édition Les Humanoïdes Associés tentent de publier Gen d’Hiroshima mais le public n’est pas au rendez-vous. 
\paragraph{}
En 1988, Jacques Glénat rentre du Japon avec un manga intitulé Akira qu’il juge très prometteur. Il apparait d’abord en 1990 sous forme de fascicule couleurs mais ne rencontre pas le succès. En 1991, la sortie du film tiré de ce manga va éveiller un vif intérêt des français envers l’art des mangakas. Glénat publie ensuite Akira dans un format livre de poche ayant ainsi un grand succès. 
\paragraph{}
Mais les mangas en version papier connaissent un véritable succès seulement avec la sortie de Dragon Ball en 1993. Suite à ce succès, en 1995, Casterman lance sa collection manga et Dargaud crée sa filiale Kana.  Il existe 	aujourd’hui trente-trois éditeur spécialisé en France.

\section{Les films d'animation}
\paragraph{}
Le plus grand honneur pour un mangaka c’est de voir son manga adapté en anime puis en film d’animation. Le premier film d’animation tiré d’un manga qui est sorti en France fut Akira en 1991. Puis le film Ghost In The Shell de Mamoru Oshii sort en 1997. Sa suite, Innocence, est sélectionnée au festival de Cannes en 2004.
\paragraph{}
En 1995, Porco Rosso de Hayao Miyazaki sort au cinéma en France. Mais c’est en 2000 que Miyazaki conquis le public français avec Princesse Mononoké. Suite au triomphe du Voyage de Chihiro, les anciens films de Miyazaki sont diffusés en France. 
\paragraph{}
En 2003, sort Interstella 5555 qui est une mise en scène de l’album « Discovery » de Daft Punk par Leiji Matsumoto le créateur d’Albator.

\section{Les raisons d'un tel succès}
\paragraph{}
A l’origine, les œuvres japonaises ont été introduites en France pour satisfaire un public d’enfants. Cependant, elles ont très vite intéressées les adolescents. 
\paragraph{}
Cet engouement peut s’expliquer par un tout nouveau type de réalisation : la mise en scène est dynamique et cinématographique. En effet, il y a une alternance continue entre les différents types de cadrages tels que des gros plan sur les visages, les plans larges pour découvrir le décor, des mouvements de caméra et la technique du face à face entre deux personnages présentant sur le même plan le dos d’un personnage et la face d’un deuxième personnage. 
\paragraph{}
De plus, les scénarios innovants rendent les mangas très attrayant. En effet, les séries se suivent comme des feuilletons ce qui permet de développer des scénarios complexes. Les personnages ont des psychologies très développées et qui évolue tout au long de la série suite aux expériences qu’ils traversent. Enfin, il arrive qu’il y ait des flashbacks lorsqu’un personnage se remémore un évènement ce qui permet aux spectateurs ou lecteurs de comprendre l’histoire même s’ils ont loupés un épisode. 
\paragraph{}
Les mangas sont également adaptés au public visé : il existe des Shônen pour les jeunes garçons qui mettent en scène des affrontements entre les personnages, des Shôjo pour les jeunes filles qui regroupent des histoires de romances, de jeunes filles aux pouvoirs magiques et de jeunes sportives. Ce genre eut un véritable succès grâce à un manque de concurrence au sein des bandes dessinées belges. En effet, les thèmes abordés sont nouveaux, les personnages sont souvent des adolescents et le rythme de publication soutenu permet au personnage de vieillir avec le lecteur ce qui permet au lecteur de s’identifier au personnage. Il existe d’autres genres tels que les Seinen ou les Josei destiné aux jeunes adultes et dont les thèmes sont plus matures et la violence plus crue.
\paragraph{}
L’aspect de chaque personnage est étudié dans les moindres détails : les yeux sont toujours assez grands pour être très expressifs et les chevelures sont d’un style séduisant mais différent pour permettre un attachement à celui-ci. De plus, les tenues sont plus ou moins réfléchies suivant le contexte de l’histoire : dans un lycée, les personnages porteront un uniforme tandis que dans une série historique ou fantastique, les tenues seront réfléchies au détail près. 
\paragraph{}
Enfin, l’impression en noir et blanc et le format poche donne un style épuré et un coût faible à l’œuvre. 

\section{Des fans impliqués}
\paragraph{}
Dans les années 90 apparait des fanzines autrement dit des magazines créés par des fans pour des fans. En effet, en 1991 a lieu la première parution d’AnimeLand. Il est encore en activité aujourd’hui. En 2006, les Humanoïdes Associés crée le Shogun Mag, un magazine de prépublication comme il en existe au Japon. 
\paragraph{}
Internet a aussi permis aux mangas de se propager. En effet, les fans ont créés les fansubs et les scantrads. Le fansub consiste à se regrouper en « team » pour traduire et sous titrer les animes. Le scantrad, quant-à-lui, consiste à scanner des mangas papier pour ensuite les traduire. Bien qu’étant illégal, cette pratique concerne principalement des séries non licenciés en France ce qui permet de promouvoir et d’évaluer la popularité potentielle de celles-ci.

\section{Un Japon qui fascine et qui inspire}
\paragraph{}
Les mangas permettent aussi de découvrir la culture japonaise. En effet, les mangas ont tendances à décrire certains aspects de la culture japonaise. 
\paragraph{}
En 1999, la Japan Expo est créée. Il s’agit d’un festival regroupant des stands sur les mangas, les animes, les jeux vidéo, le cinéma, la J-Music et la culture japonaise. Il permet aussi aux fans de faire du cosplay, une activité qui consiste à se déguiser comme son personnage favori du moment. 
\paragraph{}
L’art du manga fascine aussi les auteurs de bande dessinée français. En effet, en 2005, Jean Giraud collabore avec Jirô Taniguchi pour créer Icare. De plus, on assiste à l’apparition de « manfra », mangas français, tels que Dofus ou DreamLand.

\chapter{Un vocabulaire}

\chapter{Les produits dérivés}

\chapter{Les conventions}
\section{L'origine des conventions}
\paragraph{}
Les rencontres dédiées aux mangas et anime sont anciennes au Japon. Une pionnière du genre est Comiket, créée en 1975 à Tokyo dans le but de populariser des d\=ojinshi, mangas créés et publiés par des fans. Elles se multiplient au fil des années avec l’aide des éditeurs, qui y voient un bon moyen d’assurer la publicité de leurs séries.
\paragraph{}
Le phénomène se développe également en Occident dans les années 1980 avec l’importation de la culture japonaise. Aux États-Unis, ces anime conventions rejoignent les festivals de culture populaire américaine : science-fiction, comic books, etc.. Leur véritable succès ne commencera toutefois que dans les années 1990, décennie durant laquelle de tels évènements apparaissent aussi en Europe. En France, la plus célèbre est Japan Expo, créée en 1999, qui lance pour de bon le phénomène des conventions françaises (l’anglicisme convention ayant été adopté pour les différencier des expositions plus classiques). D’autres évènements suivent la tendance : Paris Manga, Japan Expo Sud… Leur point commun est que leur croissance rapide les a poussés à s’ouvrir à une culture geek plus générale, invitant des créateurs de séries, jeux vidéo, films de SF, etc.

\section{Le contenu des conventions}
\paragraph{}
Les invités d’honneur sont au centre de ces évènements : il peut s’agir d’un mangaka, d’un scénariste ou d’un designer célèbre présent pour des dédicaces ou une conférence. Ils peuvent également tenir un atelier. De nombreux autres artistes tiennent des stands de dédicace. Ceux-ci sont parfois tenus par un éditeur occidental, qui compte sur la présence des artistes pour faire découvrir une série en vue d’une publication.
\paragraph{}
On y trouve aussi de nombreux stands de vente, où sont proposés des mangas mais également de nombreux produits dérivés : peluches, posters, et même épées. Il est rare de revenir d’une convention sans un ou deux goodies.
\paragraph{}
Les salles de conférence sont également employées pour des avant-premières et projections tests, souvent suivies de panels où des artistes répondent aux questions des fans. Des concours y ont également lieu, notamment de cosplay, discipline chère aux conventions. Les fans y trouvent un moyen d’exprimer leur créativité en confectionnant des tenues inspirées ou directement reprises de celles de personnages de mangas ou jeux vidéo. Le phénomène est sans doute celui dont la croissance a été la plus sensible ces dernières années, parfois au regard moqueur des médias généralistes.
\paragraph{}
Enfin, les conventions proposent des expositions : d’art japonais, de dessins préparatoires, etc.